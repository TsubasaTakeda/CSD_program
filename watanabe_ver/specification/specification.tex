\documentclass[a4paper,dvipdfmx]{ujarticle}

\usepackage[top=40truemm,bottom=40truemm,left=30truemm,right=30truemm]{geometry}
\usepackage{amsmath,amssymb}
\usepackage{graphicx}
\usepackage{color}
\usepackage{ascmac}
\usepackage{bm}
\usepackage{float}
\usepackage{framed}
\usepackage{algorithm}
\usepackage{algorithmic}
\usepackage{rmss_symbol}
\usepackage{url}
\usepackage{hyperref}
\usepackage{pxjahyper}


\newtheorem{definition}{定義}
\newtheorem{proposition}{命題}
\newtheorem{theorem}{定理}
\newtheorem{lemma}{補題}
\newtheorem{assumption}{仮定}


\begin{document}


\vspace{1zh}
\begin{center}
\LARGE{

    渡邊さんに合わせたプログラム

}
\end{center}
\vspace{1zh}
\begin{flushright}

    作成日:2021年6月15日\\
    交通制御学研究室 武田翼

\end{flushright}
\vspace{1zh}



{\normalsize %
\begin{center}%
\begin{minipage}{.8\textwidth}%
\vspace{1.0em} %
%
\begin{center}
\textbf{概要}
\end{center}


    渡邊さんに合わせたプログラム作成用メモ.


\end{minipage}%
\end{center}%
}%


\section{全体像}


    全体としての入力と出力は以下の通り.
    \begin{itemize}
        \item \textbf{入力}:
            \begin{itemize}
                \item ノード数
                \item タスク起点数
                \item 平均ドライバー数
                \item 各種パラメータ
            \end{itemize}

        \item \textbf{出力}:
            \begin{itemize}
                \item 計算時間
                \item 目的関数値誤差
            \end{itemize}
    \end{itemize}


    全体の流れ(必要なモジュール)は以下の通り.
    ただし,\textcolor{blue}{青}はすでにあるプログラム.
    \textcolor{red}{赤}はすでにあるが修正 or 作成が必要なプログラム.

    \textbf{データ生成}
    \begin{enumerate}
        \item \textcolor{blue}{ネットワーク生成}
            \begin{itemize}
                \item \textbf{入力}:ノード数
                \item \textbf{出力}:各ノード間の最短経路コスト行列
                \item ネットワーク生成(ノード数に合わせたランダムネット)
                \item 各ノード間の最短経路コスト行列の作成(Dijkstra)
            \end{itemize}

        \item \textcolor{blue}{タスク起点生成}
            \begin{itemize}
                \item \textbf{入力}:ノード数,タスク起点数
                \item \textbf{出力}:タスク起点ノードベクトル.
                \item 全ノードからランダムに選ぶだけ.
            \end{itemize}

        \item \textcolor{red}{荷主情報の生成}
            \begin{itemize}
                \item \textbf{入力}:ノード数,タスク起点数,平均ドライバー数
                \item \textbf{出力}:荷主分布ベクトル
                \item ランダムに配置するだけ.
            \end{itemize}


        \item \textcolor{red}{ドライバー情報の生成}
            \begin{itemize}
                \item \textbf{入力}:ノード数,タスク起点数,平均ドライバー数,LOGITパラメータ
                \item \textbf{出力}:ドライバー分布行列,個人ドライバーのコスト行列
                \item ドライバーをランダムに配置
                \item 個人ドライバーのコスト行列作成
            \end{itemize}
    \end{enumerate}

    \textbf{マッチング計算}
    \begin{enumerate}
        \item \textcolor{red}{加速勾配法}
            \begin{itemize}
                \item \textbf{入力}:ノード間最短経路コスト行列,荷主分布ベクトル,ドライバー分布ベクトル
                \item \textbf{出力}:最適価格
                \item 加速勾配法で最適化問題を解く.
            \end{itemize}

        \item \textcolor{red}{最適配分(実数値)を生成}
            \begin{itemize}
                \item \textbf{入力}:最適価格
                \item \textbf{出力}:最適配分(実数値)
                \item 仮想ネットワーク上のMarkov連鎖配分で最適配分を計算.
            \end{itemize}

        \item \textcolor{red}{最適配分(整数値)を生成}
            \begin{itemize}
                \item \textbf{入力}:最適配分(実数値)
                \item \textbf{出力}:最適配分(整数値)
                \item 適当に丸める.
            \end{itemize}

        \item \textcolor{red}{マッチング計算(ワンちゃん渡邊さんのでいけるかも)}
            \begin{itemize}
                \item \textbf{入力}:ドライバーコスト行列のみver,and +最適配分(整数値)ver
                \item \textbf{出力}:個別マッチング行列
                \item 線形計画問題を解く.
            \end{itemize}
    \end{enumerate}





\end{document}